\PassOptionsToPackage{unicode=true}{hyperref} % options for packages loaded elsewhere
\PassOptionsToPackage{hyphens}{url}
%
\documentclass[]{article}
\usepackage{lmodern}
\usepackage{amssymb,amsmath}
\usepackage{ifxetex,ifluatex}
\usepackage{fixltx2e} % provides \textsubscript
\ifnum 0\ifxetex 1\fi\ifluatex 1\fi=0 % if pdftex
  \usepackage[T1]{fontenc}
  \usepackage[utf8]{inputenc}
  \usepackage{textcomp} % provides euro and other symbols
\else % if luatex or xelatex
  \usepackage{unicode-math}
  \defaultfontfeatures{Ligatures=TeX,Scale=MatchLowercase}
\fi
% use upquote if available, for straight quotes in verbatim environments
\IfFileExists{upquote.sty}{\usepackage{upquote}}{}
% use microtype if available
\IfFileExists{microtype.sty}{%
\usepackage[]{microtype}
\UseMicrotypeSet[protrusion]{basicmath} % disable protrusion for tt fonts
}{}
\IfFileExists{parskip.sty}{%
\usepackage{parskip}
}{% else
\setlength{\parindent}{0pt}
\setlength{\parskip}{6pt plus 2pt minus 1pt}
}
\usepackage{hyperref}
\hypersetup{
            pdftitle={COVID-19 in Russia, daily report},
            pdfborder={0 0 0},
            breaklinks=true}
\urlstyle{same}  % don't use monospace font for urls
\usepackage[margin=1in]{geometry}
\usepackage{longtable,booktabs}
% Fix footnotes in tables (requires footnote package)
\IfFileExists{footnote.sty}{\usepackage{footnote}\makesavenoteenv{longtable}}{}
\usepackage{graphicx,grffile}
\makeatletter
\def\maxwidth{\ifdim\Gin@nat@width>\linewidth\linewidth\else\Gin@nat@width\fi}
\def\maxheight{\ifdim\Gin@nat@height>\textheight\textheight\else\Gin@nat@height\fi}
\makeatother
% Scale images if necessary, so that they will not overflow the page
% margins by default, and it is still possible to overwrite the defaults
% using explicit options in \includegraphics[width, height, ...]{}
\setkeys{Gin}{width=\maxwidth,height=\maxheight,keepaspectratio}
\setlength{\emergencystretch}{3em}  % prevent overfull lines
\providecommand{\tightlist}{%
  \setlength{\itemsep}{0pt}\setlength{\parskip}{0pt}}
\setcounter{secnumdepth}{0}
% Redefines (sub)paragraphs to behave more like sections
\ifx\paragraph\undefined\else
\let\oldparagraph\paragraph
\renewcommand{\paragraph}[1]{\oldparagraph{#1}\mbox{}}
\fi
\ifx\subparagraph\undefined\else
\let\oldsubparagraph\subparagraph
\renewcommand{\subparagraph}[1]{\oldsubparagraph{#1}\mbox{}}
\fi

% set default figure placement to htbp
\makeatletter
\def\fps@figure{htbp}
\makeatother


\title{COVID-19 in Russia, daily report}
\author{}
\date{\vspace{-2.5em}}

\begin{document}
\maketitle

\textbf{Коронавирус в России.} Графический update на \textbf{2021-08-24
11:30:00}. Суммарное число зарегистрированных случаев по России c
момента начала эпидемии \textasciitilde{} \textbf{6785.4} тыс. Прирост
за сутки -- \textbf{18.8} тыс. Прирост связанных с COVID смертей за
сутки: \textbf{794}, всего с начала эпидемии в оперативных сводках
зарегистрировано смертей: \textbf{177638}. Суммарный прирост новых
случаев за последние две недели на 100 тыс. населения -- \textbf{200.4}.
Суммарное по Москве -- \textbf{1557.4} тыс., по Петербургу --
\textbf{568.5} тыс.

\textbf{Лидеры по абсолютному приросту за сутки} (\textbf{265} и более
случаев):

\begin{longtable}[]{@{}l@{}}
\toprule
x\tabularnewline
\midrule
\endhead
Московская область : 1125\tabularnewline
Москва : 1105\tabularnewline
Санкт-Петербург : 600\tabularnewline
Свердловская область : 521\tabularnewline
Ростовская область : 483\tabularnewline
Пермский край : 474\tabularnewline
Воронежская область : 463\tabularnewline
Красноярский край : 445\tabularnewline
Самарская область : 434\tabularnewline
Нижегородская область : 427\tabularnewline
Иркутская область : 395\tabularnewline
Омская область : 391\tabularnewline
Челябинская область : 377\tabularnewline
Республика Крым : 349\tabularnewline
Ставропольский край : 349\tabularnewline
Волгоградская область : 349\tabularnewline
Оренбургская область : 347\tabularnewline
Хабаровский край : 312\tabularnewline
Республика Башкортостан : 300\tabularnewline
Ульяновская область : 290\tabularnewline
Астраханская область : 286\tabularnewline
Архангельская область : 265\tabularnewline
\bottomrule
\end{longtable}

\textbf{Техническое.} В виду многочисленности графиков по регионам,
отсылаю за ними к папкам в репозитории.

\begin{itemize}
\item
  \textbf{Графики по регионам:} \textless{}
  \url{https://github.com/alexei-kouprianov/COVID.2019.ru/tree/master/plots/regions}
  \textgreater{}
\item
  \textbf{Скрипт и данные.} \textless{}
  \url{https://github.com/alexei-kouprianov/COVID.2019.ru}
  \textgreater{}
\item
  \textbf{Исходные данные.} \textless{}
  \url{https://стопкоронавирус.рф/information/} \textgreater{}
\end{itemize}

\end{document}
